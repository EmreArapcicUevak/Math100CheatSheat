\documentclass[a4paper, 15pt]{article}

\usepackage[margin = 1in]{geometry} % for spacing around
\usepackage{graphicx} % for including images in your pdfs
\usepackage{xcolor} % for including colors in your pdf
\usepackage{soul} % for text decoration
\usepackage[utf8]{inputenc} % for encoded text
\usepackage[T1]{fontenc}
\usepackage{setspace} % for setting different line spacings between paragrafs.
\usepackage{enumerate} % for letting us get more detailed enumerate lists
\usepackage{multirow} % to let us combine more rows together
\usepackage{colortbl} % for decorating tables
\usepackage{amsmath} % used for representing more complicated math displays
\usepackage{supertabular}
\usepackage{longtable} % both of these packages are used to making really big tables
\usepackage{wrapfig} % allows us to wrap text around figures
\usepackage{fancyhdr} % for making fancy headers
%\usepackage{bibtex} % for making better bibliographies
\usepackage[pdftex]{hyperref} % for letting us make links
\usepackage{lscape} % Allows us to flip from portrait to landspace
\usepackage{tikz} % for high detailed drawing
\usepackage{multicol} % To put things side by side
\usepackage{rotating} % For rotating objects
% \usepackage{draftwatermark} % For adding watermarks
\usepackage{MnSymbol} % for using multiple symbols
\usepackage{mathtools} % Used for more math symbols
\usepackage{xfrac} % For more complciated fractions and to add derivitives
\usepackage{hyperref} % for hyper links
\usepackage{enumitem} % for better enum lists
\usepackage{tcolorbox} % for adding colored text boxes
\usepackage{bm} % Adding bold text to math inputs

% Setting up the default image path
\graphicspath{{./Images/}}

% Implementing authro details
\title{Math 100 Cheat Sheet}
\author{Emre Arapcic-Uvak}
\date{}

% Setting up the fancy page style
\fancypagestyle{customStyle}{
	\lhead{} \chead{} \rhead{}
	\lfoot{} \cfoot{\thepage} \rfoot{}
	\renewcommand{\headrulewidth}{0pt}
	\renewcommand{\footrulewidth}{1pt}
}
\pagestyle{customStyle}

% Setting up hyperref options
\hypersetup {
	colorlinks = false,
	citecolor = black,
	filecolor = blue,
	linkcolor = blue,
	urlcolor = blue,
	pdftex
}

% Custom commands
\newcommand{\importantStar}{
	\begin{Large}
		\textcolor{red}{$\filledstar$}
	\end{Large}
}

\begin{document}
	\maketitle
	\vspace{5mm}
	
	\begin{abstract}
		\begin{center}
			\noindent This document contains a set of identities and equations for Math100
		\end{center}
	\end{abstract}
	\pagebreak
	
	\tableofcontents
	\pagebreak
	
	\section{Logarithms}
		\subsection{General definition of a logarithm}
			\noindent Logarithms are inverse function of exponential function meaning that if we want to remove an exponent we can apply an logarithm
			
			\begin{equation*}
				\log_{b}(a) = c \implies b^c = a
			\end{equation*}
		
			\subsubsection{Example}
				\noindent Lets say that we want to figure out 2 to the power of what number gives us 524288, or in other words $2^x = 524288$. Well we can approach this problem in two ways, the first way is to apply the definition of the logarithm and phase the problem as such $x = \log_{2}(524288)$. Or we can do this by placing the both sides of the equation into logarithms as follows $\log_{2}(2^x) = \log_{2}(524288) \implies x = \log_{2}(524288)$.
				
				
			\subsubsection{Domain}
				\[\log_{b}(a) = c\]
				\noindent The logarithm above is valid for the following values:
				\[a \in (0, +\infty)\]
				\[b \in (0,1) \cup (1, +\infty)\]
		\subsection{Logarithm Identities}
			\noindent These are some essential logarithm identities that can be found in many math books:
			\begin{enumerate}
				\item $\ln(a) = \log_{e}(a)$
				\item $\log(a) = \log_{10}(a)$
			\end{enumerate}
		\subsection{Logarithm Rules}
			\noindent There are a lot of rules that can help us when we are dealing with logarithms, some of these rules are:
			\begin{enumerate}
				\item $\log_{b}(1) = 0$
				\item $\log_{b}(b) = 1$
				\item $\log_{b}(b^x) = x$
				\item $\log_{b}(a^c) = c*\log_{b}(a)$
				\item $\log_{b}(a) + \log_{b}(c) = \log_{b}(a*c)$ \hspace{4mm} \importantStar
				\item $\log_{b}(a) - \log_{b}(c) = \log_{b}(\frac{a}{c})$ \hspace{4mm} \importantStar
				\item $\log_{b}(a) = \frac{\log_{c}(a)}{\log_{c}(b)}$ \hspace{4mm} \importantStar
			\end{enumerate}
\end{document}
